\section*{摘要}
本研究深入探讨了人工智能(AI)\cite{casal2023ai}技术的演进历程、当前发展前沿及其对全球社会与经济产生的深远影响。
自20世纪中叶概念提出以来,
人工智能经历了数次发展浪潮,尤其是在近年来,得益于计算能力的指数级提升、大数据资源的积累以及机器学习算法的创新,
人工智能已取得了突破性的进展。本文首先回顾了人工智能从符号主义到连接主义的理论演变,并梳理了神经网络、
支持向量机等关键技术的发展脉络。随后,本研究着重分析了当前人工智能领域的核心突破,包括以Transformer
模型为代表的深度学习在自然语言处理和计算机视觉领域的广泛应用,生成式对抗网络(GANs)和扩散模型在内容创作中的革命性作用,
以及强化学习在复杂决策和自动化控制方面的显著成效。

此外,本文详细探讨了人工智能在各个关键行业领域的具体应用,例如在医疗健康领域用于疾病
诊断、药物研发和个性化治疗;在教育领域实现智能辅导和个性化学习路径;在金融领域优化风险管理和欺诈检测;
以及在交通领域推动自动驾驶和智能物流的发展。在肯定人工智能巨大潜力的同时,本研究也客观分析了其带来的挑战,
包括数据隐私与安全、算法偏见、就业结构变化、以及伦理道德与法律责任等复杂问题。最后,本文对人工智能的未来发展趋势进行了展望,
预示了通用人工智能(AGI)的潜在方向、人机协作模式的深化以及人工智能在可持续发展目标中的作用。研究强调,在积极拥抱人工智能带来
的机遇的同时,社会各界必须共同努力,制定健全的监管框架、推动跨学科研究、并加强公众教育,以确保人工智能技术能够负责任地、公平地
、普惠地造福全人类,构建一个更加智能、高效和可持续的未来。 