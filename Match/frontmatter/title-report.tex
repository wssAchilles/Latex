\begin{titlepage}

% 添加背景图片
\AddToShipoutPicture*{%
    \put(0,0){%
        \includegraphics[width=\paperwidth,height=\paperheight]{figure/cover.png}%
    }%
}

\centering % 居中对齐

% 在顶部添加图标和文字
\begin{tikzpicture}[remember picture, overlay]
    \node[anchor=north west, inner sep=0pt] at ([xshift=10mm,yshift=-3mm]current page.north west) {%
        \includegraphics[width=2cm]{figure/logo-black.png} % 调整logo的宽度
    };
    \node[anchor=north west, inner sep=0pt] at ([xshift=35mm,yshift=-16mm]current page.north west) {%
        {\fontsize{16}{20} \selectfont 中国大学生计算机设计大赛大数据主题赛}
    };
\end{tikzpicture}

\vspace*{2cm} % 在顶部添加垂直空间
\hspace*{0cm}
% 设置并打印标题
% {\fontsize{50}{60} \textbf{智潮云涌,变革云起} \par}
\begin{figure}[H]
    \centering
    \includegraphics[width=1.1\linewidth]{figure/title.png}
    \label{fig:title}
\end{figure}
\vspace{1cm} % 在标题和副标题之间添加垂直空间
% 向上移动文本
\vspace*{-1.2cm}
% 设置并打印副标题
% \noindent
% \hspace*{4.2cm}
% \begin{minipage}{15cm}
% {\fontsize{25}{36} \selectfont \textbf{基于}{\textcolor{deepgreen}{\textbf{DLF-LSTM}}}\textbf{的乡村发展报告}}
% \end{minipage}


\vfill % 推送后续内容到页面底部

% 在底部横向居中放置图标
\begin{tikzpicture}[remember picture, overlay]
    \node[anchor=south, inner sep=0pt] at ([yshift=90mm]current page.south) {%
        \includegraphics[width=4cm]{figure/logo2.png} % 调整logo的宽度
    };
\end{tikzpicture}

% 使用 raisebox 移动作者信息
\raisebox{9cm}{
    {\fontsize{20}{30}王雨欣 \hspace{0.3cm} 唐胡煜 \hspace{0.3cm} 许子祺 \hspace{0.3cm} 张倩 \hspace{0.3cm} 夏子城\par}
}

\end{titlepage}