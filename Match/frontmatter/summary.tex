\chapter*{摘要}
在新时代背景下,科技创新与社会变革成为推动国家战略转型和经济高质量发展的关键引擎。本文立足于教育、企业、就业、国际影响力等多个维度,借助 SARIMA-LSTM 融合预测模型,深入剖析我国科技创新的现实基础与发展趋势,旨在为建设科技强国与推动社会深层变革提供系统性策略。

首先,教育是国家进步的基石,本文通过时间序列数据分析了我国教育经费与高校科研投入的演变趋势,研究显示,科技创新反哺教育发展,促进教育公平与质量提升,为人才结构优化与原始创新能力增强奠定基础。

在企业发展维度,本文从企业技术发展与资本支持角度出发,采用时间序列建模与趋势预测方法,分析了风险投资与独角兽企业的发展路径,得出企业创新能力显著增强、区域创新生态日益完善的结论。这与国家推动战略性新兴产业发展、促进区域协同创新的目标高度一致。

就业结构方面,本文从劳动市场变化与岗位结构演化出发,利用 LSTM 模型对就业趋势进行预测,得出服务业和高技能岗位占比持续上升、传统岗位向智能化转型的结论。该趋势与构建现代人力资源体系、推动高质量充分就业的国家战略目标相契合。

最后,在探讨国际影响力时,本文从全球创新竞争格局出发,结合创新指数、高技术产品出口与 AI 治理透明度等指标,构建模型分析其演变趋势,得出中国科技国际竞争力持续上升、新兴技术出口快速增长的结论,这使我国加快了从“科技追赶者”向“全球引领者”转型的步伐。



\textbf{关键词:SARIMA-LSTM\hspace{0.5cm}斯皮尔曼相关性\hspace{0.5cm}线性回归\hspace{0.5cm}PCA
\hspace{0.5cm}创新指数 } 