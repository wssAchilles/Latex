\chapter*{摘要}
在新时代背景下,乡村发展不仅是实现乡村振兴战略的关键,更是推动社会和谐与共同富裕的重要途径。本文从多个维度探究乡村发展的多方面内容,包括粮食安全、科技影响、乡村建设、产业结构、乡村就业和人口等方面,旨在提出全面而具体的发展策略。

首先,粮食安全是国家的生命线,保障粮食安全是国家战略的基石。本文通过对近年来的粮食播种面积和主要粮食产量的时间序列分析,突显了我国在确保粮食安全方面取得的显著成就。

在科技影响方面,本文从农业生产和农民生活质量两个层面进行探讨。利用Holt指数平滑预测法,展示了科技创新如何为农业生产注入活力,并显著提升农民的生活水平,这也与我国发布的数字农业发展报告相呼应。

乡村建设部分,本文利用复合年增长率(CAGR)深入分析了各省基础设施建设的进展,并提出了针对性的建议。同时,通过量化数据展示了生态环境保护和改善对乡村环境的积极影响。这些努力最终促进了乡村旅游业的发展,与中央农村工作会议提出的美丽乡村建设目标相契合。

在探讨农民生活水平时,本文从农村就业和产业结构变化出发,使用循环神经网络预测了乡村发展的潜在产业方向。通过斯皮尔曼分析收入支出信息并且进行PCA和牛顿多元回归、分析恩格尔系数以及基本生活工具,深入了解了乡村居民的生活质量。

最后,在城乡人口方面,本文通过DLF-LSTM预测分析指出,随着中国总人口趋于饱和,城市人口稳定,现在正是推动乡村发展、促进农村人口回流的大好时机。国家领导人在推动乡村振兴中的战略思考中,也强调了加快乡村建设,实现城乡融合发展的重要性。

\textbf{关键词:DLF-LSTM\hspace{0.5cm}斯皮尔曼相关性\hspace{0.5cm}牛顿多元回归\hspace{0.5cm}PCA\hspace{0.5cm}CARG\hspace{0.5cm}Holt} 